\documentclass[11pt,a4paper]{scrreprt}

\usepackage{cmap}
\usepackage[T1]{fontenc} 
\usepackage[utf8]{inputenc}
\usepackage[english,russian]{babel}
\usepackage{ulem}
\usepackage{array}
\usepackage{setspace}
\usepackage{amsmath}
\usepackage{enumitem}

\usepackage{geometry}
\geometry{left=30mm}
\geometry{right=15mm}
\geometry{top=20mm}
\geometry{bottom=20mm}
\geometry{foot=.7cm}

\emergencystretch 10em

\newenvironment{signstabular}[1][1]{
	\renewcommand*{\arraystretch}{#1}
	\tabular
}{
	\endtabular
}

\newcommand*{\undertext}[2]{%
	\begin{tabular}[t]{@{}c@{}}%
		#1\\\relax(\scriptsize#2)%
	\end{tabular}
}

\begin{document}
\thispagestyle{empty}
\begin{center}
	\fontsize{11pt}{0.3\baselineskip}\selectfont \textbf{Министерство науки и высшего образования Российской Федерации \\ Федеральное государственное бюджетное образовательное учреждение \\ высшего образования \\ <<Московский государственный технический университет имени Н.Э. Баумана \\ (национальный исследовательский университет)>> \\ (МГТУ им. Н.Э. Баумана)}

	\makebox[\linewidth]{\rule{\textwidth}{1pt}}
	\makebox[\linewidth]{\rule{\textwidth}{2.5pt}}
	
	\vspace{\baselineskip}

\end{center}

\begin{flushright}
	\begin{minipage}{0.33\linewidth}
		\fontsize{11pt}{0.5\baselineskip}\selectfont
		УТВЕРЖДАЮ \\
		Заведующий кафедрой \undertext{\uline{\hfill ИУ7\hfill}}{Индекс}\\
		\uline{\hphantom{~~И. В.Руда~}} \undertext{\uline{~И. В.Рудаков~}}{И.О.Фамилия}
		<<\uline{\hphantom{~~~~~~~~~~}}>> \uline{\hphantom{~~~~~~~~~~~~~~~~~~}} 2022 г.
	\end{minipage}
\end{flushright}
\vspace{1pt}
\begin{center}
	\fontsize{18pt}{\baselineskip}\selectfont \textbf{З А Д А Н И Е}\\
	\fontsize{16pt}{\baselineskip}\selectfont \textbf{на выполнение курсовой работы}
\end{center}

\normalsize
	
\begingroup
\fontsize{11pt}{1.2\baselineskip}\selectfont
\setlength{\parskip}{0.3em}
\setlength{\parindent}{0em}
по дисциплине \uline{\hfill Базы данных \hfill} \\
Студент группы \uline{\hfill ИУ7-66Б \hfill} \\ 
\uline{\centerline{\undertext{Казаева Татьяна Алексеевна}{Фамилия, имя, отчество}}} \\
\fontsize{11pt}{\baselineskip}\selectfont
Тема курсовой работы \uline{~~~Разработка базы данных для хранения и аналитики результатов статистического опроса \hfill} \\
Направленность КР (учебная, исследовательская, практическая, производственная, др.) \uline{\centerline{учебная}\hfill} \\
Источник тематики (кафедра, предприятие, НИР) \uline{\hfill кафедра \hfill} 

График выполнения работы: 25\% к \uline{6} нед., 50\% к \uline{9} нед., 75\% к \uline{13} нед., 100\% к \uline{16} нед.

\textit{\textbf{Задание:}} \uline{Спроектировать и реализовать базу данных, содержащую данные, полученные при анкетировании экспертов о тональности прилагательных. Разработать систему для разметки интерфейса.\hfill} \\
\textit{\textbf{Оформление курсовой работы:}}
\begin{enumerate}[label=\arabic*),wide=0pt]
	\item Расчетно-пояснительная записка на \uline{25-30} листах формата А4. \\
	\uline{Расчетно-пояснительная записка должна содержать постановку задачи, введение, аналитическую
	часть, конструкторскую часть, технологическую часть, экспериментально-исследовательский
	раздел, заключение, список литературы, приложения.\hfill}
	\item Перечень графического (иллюстративного) материала (чертежи, плакаты, слайды и т.д.) \\
	\uline{На защиту проекта должна быть предоставлена презентация, состоящая из 15-20 слайдов. На
		слайдах должны быть отражены: постановка задачи, использованные методы и алгоритмы,
		структура комплекса программ, интерфейс, результаты проведенных
		исследований. \hfill}
\end{enumerate}
~\\
Дата выдачи задания
<<\uline{\mbox{\hspace*{5mm}}}>> \uline{\mbox{\hspace*{2.5cm}}} 20\uline{22} г. \\
~\\
\textbf{Руководитель курсовой работы} \hfill \uline{\undertext{\mbox{\hspace*{3cm}}}{Подпись, дата}} 
\uline{\undertext{~Ю. В. Строганов~}{И.О. Фамилия}}\\
\textbf{Студент} \hfill
\uline{\undertext{\mbox{\hspace*{3cm}}}{Подпись, дата}} 
\uline{\undertext{~~~Т. А. Казаева~~~}{И.О. Фамилия}}\\
\uline{Примечание:} Задание оформляется в двух экземплярах: один выдается студенту, второй хранится
на кафедре.
\end{document}

