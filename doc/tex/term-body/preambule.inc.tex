\usepackage{cmap}
\usepackage[T1]{fontenc} 
\usepackage[utf8]{inputenc}
\usepackage[english,russian]{babel}

\usepackage{caption}
\usepackage{subcaption}
\usepackage[normalem]{ulem}

%\usepackage{soulutf8}

\usepackage{float}

\usepackage{enumitem}

\usepackage{graphicx}
\usepackage{multirow}


\usepackage{pgfplots}
\pgfplotsset{compat=newest}
\usepgfplotslibrary{units}


\usepackage{caption}
\captionsetup{labelsep=endash}
\captionsetup[figure]{name={Рисунок}}
\captionsetup[subfigure]{name={Рисунок}}
\captionsetup[subtable]{labelformat=simple}
\captionsetup[subfigure]{labelformat=simple}
\renewcommand{\thesubtable}{\text{Таблица }\arabic{chapter}\text{.}\arabic{table}\text{.}\arabic{subtable}\text{ --}}
%\renewcommand{\thesubfigure}{\arabic{chapter}\text{.}\arabic{figure}\text{.}\azbuk{\subfigure}\text{ --}}


\usepackage{textcomp}
\usepackage{chngcntr}

\usepackage{amsmath}
\usepackage{amsfonts}
\usepackage{array}

\usepackage{geometry}
\geometry{left=30mm}
\geometry{right=15mm}
\geometry{top=20mm}
\geometry{bottom=20mm}
\geometry{foot=1.7cm}

\usepackage{titlesec}
\titleformat{\section}
{\normalsize\bfseries}
{\thesection}
{1em}{}
\titlespacing*{\chapter}{0pt}{-30pt}{8pt}
\titlespacing*{\section}{\parindent}{*4}{*4}
\titlespacing*{\subsection}{\parindent}{*4}{*4}
\titlespacing*{\subsubsection}{\parindent}{*4}{*4}

%\renewcommand\thesubfloat{(\roman{subfloat})}
\renewcommand\thesubfigure{(\asbuk{subfigure})}

% Маркировка для списков
\def\labelitemi{$\circ$}
\def\labelitemii{$*$}
\usepackage{pdflscape}

\usepackage{setspace}
\onehalfspacing % Полуторный интервал

\captionsetup[table]{skip=0pt,singlelinecheck=off, justification=raggedleft}
\captionsetup[table]{skip=0pt,singlelinecheck=off, justification=centering}

\frenchspacing
\usepackage{indentfirst} % Красная строка

\usepackage{titlesec}
\usepackage{xcolor}
% Названия глав
\titleformat{\section}{\normalsize\textmd}{\thesection}{1em}{}

\definecolor{gray35}{gray}{0.35}

\titleformat{\chapter}[hang]{\huge}{\textcolor{gray35}{\thechapter. }}{0pt}{\huge\scshape}

\titleformat{\section}{\Large}{\textcolor{gray35}\thesection}{20pt}{\Large\scshape}
\titleformat{\subsection}{\large}{\thesubsection}{20pt}{\large\scshape}
\titleformat{\subsubsection}{\large}{\thesubsubsection}{20pt}{\large\scshape}

\newcommand*{\undertext}[2]{%
	\begin{tabular}[t]{@{}c@{}}%
		#1\\\relax\scriptsize(#2)%
	\end{tabular}
}

\emergencystretch 10em

% Настройки введения

\addtocontents{toc}{\setcounter{tocdepth}{3}}
\addtocontents{toc}{\setcounter{secnumdepth}{3}}

\usepackage{tocloft,lipsum,pgffor}

\addtocontents{toc}{~\hfill\textnormal{Страница}\par}

\renewcommand{\cftpartfont}{\normalfont\textmd}

\addto\captionsrussian{\renewcommand{\contentsname}{Содержание}}
\renewcommand{\cfttoctitlefont}{\Huge\textmd}

\renewcommand{\cftchapfont}{\normalfont\normalsize}
\renewcommand{\cftsecfont}{\normalfont\normalsize}
\renewcommand{\cftsubsecfont}{\normalfont\normalsize}
\renewcommand{\cftsubsubsecfont}{\normalfont\normalsize}

\renewcommand{\cftchapleader}{\cftdotfill{\cftdotsep}}

\usepackage{listings}
\usepackage{pdflscape}
\usepackage{everypage}
\usepackage{xcolor}

%\bibliographystyle{gost780u.bst}
%
%\usepackage[backend=biber,
%%			bibencoding=utf8,
%%			sorting=nyt,
%%			maxcitenames=2,
%			style=gost-numeric-min,
%%			autolang=other, 
%%			natbib=true,
%%			maxnames=99,
%			uniquename=false]{biblatex}

\usepackage{csquotes} 
\usepackage[backend=biber,
			style=gost-numeric,
			maxcitenames=3,
			maxbibnames=12,
			minnames=1,
			movenames=false,
			ibidtracker=false,
			sorting=none,
			autolang=other]{biblatex}
			
\DeclareSourcemap{
	\maps[datatype=bibtex]{
		\map{
			\step[fieldsource=langid, match=russian, final]
			\step[fieldset=presort, fieldvalue={a}]
		}
		\map{
			\step[fieldsource=langid, notmatch=russian, final]
			\step[fieldset=presort, fieldvalue={z}]
		}
	}
}

\addbibresource{ref-lib.bib} % База библиографии

\usepackage[pdftex]{hyperref} % Гиперссылки
\hypersetup{hidelinks}

% Листинги 
\usepackage{listings}

\definecolor{darkgray}{gray}{0.15}

\definecolor{teal}{rgb}{0.25,0.88,0.73}
\definecolor{gray}{rgb}{0.5,0.5,0.5}
\definecolor{b-red}{rgb}{0.88,0.25,0.41}
\definecolor{royal-blue}{rgb}{0.25,0.41,0.88}



% какой то сложный кусок со стак эксчейндж для квадратных скобок
\makeatletter
\newenvironment{sqcases}{%
	\matrix@check\sqcases\env@sqcases
}{%
	\endarray\right.%
}
\def\env@sqcases{%
	\let\@ifnextchar\new@ifnextchar
	\left\lbrack
	\def\arraystretch{1.2}%
	\array{@{}l@{\quad}l@{}}%
}
\makeatother

% и для матриц
\makeatletter
\renewcommand*\env@matrix[1][\arraystretch]{%
	\edef\arraystretch{#1}%
	\hskip -\arraycolsep
	\let\@ifnextchar\new@ifnextchar
	\array{*\c@MaxMatrixCols c}}
\makeatother

\usepackage{pdflscape}
\usepackage{fancyhdr} 

\fancypagestyle{mylandscape}{
	\fancyhf{} %Clears the header/footer
	\fancyfoot{% Footer
		\makebox[\textwidth][r]{% Right
			\rlap{\hspace{.75cm}% Push out of margin by \footskip
				\smash{% Remove vertical height
					\raisebox{6in}{% Raise vertically
						\rotatebox{90}{\thepage}}}}}}% Rotate counter-clockwise
	\renewcommand{\headrulewidth}{0pt}% No header rule
	\renewcommand{\footrulewidth}{0pt}% No footer rule
}


