\addchap{ЗАКЛЮЧЕНИЕ}
Автоматизация определения тональности текста -- алгоритмически сложная задача, включающая в себя ряд не менее сложных подзадач. Одна из них -- определение характеристик субъекта тональности и выявление связи между ними и тональной оценкой. Классический подход на основе массовых опросов предоставляет более точные данные, чем подход на основе анализа общих словарей.

Данные, полученные при массовом опросе принадлежат классу Linked Object Mining. Такой набор связей является неструктурированным, следовательно, не подходит для хранения в реляционных базах данных. Графовые базы данных не накладывают ограничений на хранение такого типа хранимых данных. 

В представленной работе проведена работа с графовой базой данных с использованием СУБД Neo4j. Было проведено нагрузочное тестирование с изменением размера кучи виртуальной машины Java и контроля количества памяти, используемого для кэширования файлов. 

При корректной настройке этих параметров под нужды системы, на которой работает Neo4j скорость запросов в среднем возрастает. 

Поскольку Neo4j обладает открытым исходным кодом, перспективой исследования может быть изучение программной реализации СУБД для наиболее удачного подбора параметров конфигурации. 