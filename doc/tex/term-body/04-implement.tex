\chapter{Технологический раздел}\label{sec:impl}
\section{Выбор средств разработки}
%База данных Neo4j\cite{neo4docs} удовлетворяет требованиям, изложенным в подразделе \ref{subsec:store}, поэтому она была использована в качестве хранилища. e
В разделе \ref{subsec:store} обоснована необходимость использования графовых баз данных. В качестве системы управления базами данных была выбрана СУБД Neo4j\cite{neo4docs}, поскольку может быть доступна из ПО, написанного на практически любом языке с использованием языка запросов Cypher, через обращение к маршруту отдельным HTTP методом или через протокол <<bolt>>.

Язык Golang\cite{go} обладает достаточным набором инструментов для написания одностраничных приложений: предоставляет базовые возможности маршрутизации, интерфейс работы с базами данных и конфигурационными файлами, поэтому он был выбран в качестве реализации клиентского приложения. Базовый интерфейс маршрутизации был расширен с помощью пакета Gorilla\cite{gorilla}.

\section[Реализация клиентского приложения]{Реализация клиентского\\ приложения}
%Модельные структуры для сущностей, описывающих узлы и связи базы данных представлены на листинге \ref{lst:model}
Модельные структуры для сущностей соответствуют сущностям базы данных, представленным на рисунке \ref{fig:chen}.

На листинге \ref{lst:model1} представлена модельная структура сущности <<Аккаунт>>.
\begin{lstlisting}[label=lst:model1,caption=Модельная структура сущности <<Аккаунт>>]
type Account struct {
	UUID              string `json:"uuid"`
	Username          string `json:"username,omitempty"`
	Email             string `json:"email,omitempty"`
	Password          string `json:"password,omitempty"`
	Role              string `json:"role,omitempty"`
	EncryptedPassword string `json:"-"`
}
\end{lstlisting}
На листинге \ref{lst:model3} представлена модельная структура сущности <<Пользователь>>.
\begin{lstlisting}[label=lst:model3,caption=Модельная структура сущности <<Пользователь>>]
type User struct {
	UUID   string `json:"uuid,omitempty"`
	Age    int    `json:"age"`
	Race   string `json:"race"`
	Gender string `json:"gender"`
}
\end{lstlisting}

На листинге \ref{lst:model2} представлены модельные структуры сущностей <<Населенный пункт>>, <<Программа>> и <<Место обучения>>.
\begin{lstlisting}[label=lst:model2,caption=Модельные структуры сущностей <<Населенный пункт>>\, <<Программа>> и <<Место обучения>>]
type EducationPlace struct {
	UUID string `json:"uuid"`
	Name string `json:"name"`
}
type EducationProgram struct {
	Field string `json:"field"`
	Level string `json:"level"`
}
type Location struct {
	UUID     string `json:"uuid"`
	Name     string `json:"name"`
	Region   string `json:"region"`
	District string `json:"district"`
}
\end{lstlisting}
На листинге \ref{lst:model4} представлены модельные структуры сущностей <<Результат тестирования>> и <<Прилагательное>>.
\begin{lstlisting}[label=lst:model4,caption=Модельные структуры сущностей <<Результат тестирования>> и <<Прилагательное>>]
type Answer struct {
	Word string
	Mark int
}
type Poll struct {
	Answer []Answer
}
\end{lstlisting}

База данных хранит пароли в качестве функции свертки с модификатором входа. Хеширование и проверка пароля представлены на листинге \ref{lst:hash}.
\begin{lstlisting}[label=lst:hash,caption=Хеширование и проверка пароля]
func (a *Account) EncryptPassword() error {
	if len(a.Password) > 0 {
		enc, err := encryptString(a.Password)
		if err != nil {
			return err
		}
		a.EncryptedPassword = enc
	}
	return nil
}

func (a *Account) ComparePassword(password string) bool {
	return bcrypt.CompareHashAndPassword(
		[]byte(a.EncryptedPassword), []byte(password)) == nil
}

func encryptString(p string) (string, error) {
	b, err := bcrypt.GenerateFromPassword([]byte(p), bcrypt.MinCost)
	if err != nil {
		return "", err
	}
	return string(b), nil
}
\end{lstlisting}

Сервис, работающий с базой данных аналогичен для всех модельных структур. На листинге \ref{lst:db} представлен механизм установки соединения и регистрации пользователя.

\begin{lstlisting}[label=lst:db,caption=Установка соединения и отправка запросов]
func newDB(DBUri, DBUsername, DBPassword string) (neo4j.Driver, error) {
	db, err := neo4j.NewDriver(DBUri, neo4j.BasicAuth(DBUsername,
		 DBPassword, ""))
	if err != nil {
		return nil, err
	}
	return db, nil
}

func (r *AccountRepository) Create(a *model.Account) error {
	if err := a.Validate(); err != nil {
		return err
	}
	if err := a.EncryptPassword(); err != nil {
		return err
	}
	session := r.storage.db.NewSession(neo4j.SessionConfig{AccessMode:
		 neo4j.AccessModeWrite, DatabaseName: "goodisgood"})
	defer session.Close()
	
	if _, err := session.WriteTransaction(func(
	tx neo4j.Transaction) (interface{}, error) {
		return r.createQuery(tx, a)
	}); err != nil {
		return err
	}
	return nil
}

func (r *AccountRepository) createQuery(tx neo4j.Transaction, 
		a *model.Account) (*model.Account, error) {
	result, err := tx.Run(
	`create (a: account{id: apoc.create.uuid(),
						username: $username,
						email: $email,
						password:$password, role:"user"}) 
	return a.id as id`,
	map[string]interface{}{
		"username": a.Username,
		"email":    a.Email,
		"password": a.EncryptedPassword,
	},
	)
	
	if err != nil {
		return nil, err
	}
	
	record, err := result.Single()
	
	if err != nil {
		return nil, err
	}
	
	id, _ := record.Get("id")
	a.UUID = id.(string)
	
	return a, nil
}
\end{lstlisting}
Механизм маршрутизации запроса на авторизацию пользователя представлен на листинге \ref{lst:sessions}
\begin{lstlisting}[label=lst:sessions,caption=Маршрутизация запроса аутентификации]
func (s *server) handleSessionsCreate() http.HandlerFunc {
	type request struct {
		Username string `json:"username"`
		Email    string `json:"email"`
		Password string `json:"password"`
	}
	return func(w http.ResponseWriter, r *http.Request) {
		req := &request{}
		if err := json.NewDecoder(r.Body).Decode(req); err != nil {
			s.logger.Error(w, r, http.StatusBadRequest, err)
			return
		}
		a, err := s.ustorage.Account().FindByEmail(req.Email)
		if err != nil || !a.ComparePassword(req.Password) {
			if err != nil {
				s.logger.Infof(err.Error())
			}
			s.error(w, r, http.StatusUnauthorized, 
				errInvalidEmailOrPassword)
			return
		}
		session, err := s.sessionStorage.Get(r, sessionName)
		if err != nil {
			s.error(w, r, http.StatusInternalServerError, err)
			return
		}
		session.Values["account_id"] = a.UUID
		if err := s.sessionStorage.Save(r, w, session); err != nil {
			s.error(w, r, http.StatusInternalServerError, err)
			return
		}
		s.respond(w, r, http.StatusOK, nil)
	}
}
func (s *server) error(w http.ResponseWriter, r *http.Request,
	 code int, err error) {
	s.respond(w, r, code, map[string]string{"error": err.Error()})
}
func (s *server) respond(w http.ResponseWriter, r *http.Request,
	 code int, data interface{}) {
	w.WriteHeader(code)
	if data != nil {
		json.NewEncoder(w).Encode(data)
	}
}
\end{lstlisting}

\section{Запросы к базе данных}
Neo4j использует язык запросов Cypher\cite{cypher}. 

На листинге \ref{lst:user-find} приведен запрос связи места обучения с пользователем.
\begin{lstlisting}[label=lst:user-find,caption=Запрос связи места обучения с пользователем]
match (u:user)-[o:OWNS]->(a:account{id:$auuid})
match(e:education_place{name: $ename})
merge((u)-[s:STUDIES{field:$field, level: $level}]->(e))
\end{lstlisting}
На листинге \ref{lst:acc} приведен запрос для создания пользователя, связанного с конкретным аккаунтом. 
\begin{lstlisting}[label=lst:acc,caption=Запрос для создания пользователя\, связанного с конкретным аккаунтом]
match
	(a:account)
where
	a.id=$auuid
create (u:user{id: apoc.create.uuid(),
				race: $race,
				age: $age,
				gender:$gender})
create (u)-[o:OWNS]->(a)
return a.id as id	
\end{lstlisting}
На листинге \ref{lst:stats} приведен запрос для заполнения опроса (разметки прилагательного).
\begin{lstlisting}[label=lst:poll,caption=Запрос для заполнения опроса]
match (u:user)-[o:OWNS]->(a:account{id:$auuid})
match (w:word{name: $wname})
optional match (u)-[m:MARKED]->(w)
call apoc.do.when(
m is null,
' create (u)-[m:MARKED{mark: mrk}]->(w)
return m is not null as ok',
'return true as ok',
{u:u, w:w, mrk: $mark}) yield value
return value.ok as ok
\end{lstlisting}
На листинге \ref{lst:stats} приведен запрос для получения общей статистики.
\begin{lstlisting}[label=lst:stats,caption=Запрос для получения общей статистики]
match ()-[r:MARKED]->(w:word)
with w.name AS name, collect(r.mark) AS marklist, COUNT(r.mark) AS cnt
unwind marklist as ml
return distinct name, avg(ml) as avg
\end{lstlisting}


\section{Тестирование}
Тестирование осуществлялось с помощью стандартных механизмов Golang -- пакета Testing\cite{testing}. На листинге \ref{lst:testing} приведено тестирование функции поиска пользователя.
\begin{lstlisting}[label=lst:testing,caption=Тестирование функции поиска пользователя]
func TestAccountRepository_FindByEmail(t *testing.T) {
	d := neo4jStorage.TestDB(t, DBuri, DBUsername, DBPassword)
	s := neo4jStorage.NewStorage(d)
	e2 := "gopher@gopher.go"
	a, err := s.Account().FindByEmail(e2)
	assert.NoError(t, err)
	assert.NotNil(t, a)
}
func TestAccountRepository_Find(t *testing.T) {
	d := neo4jStorage.TestDB(t, DBuri, DBUsername, DBPassword)
	
	s := neo4jStorage.NewStorage(d)
	u1, err := s.Account().FindByEmail("gopher@gopher.go")
	if err != nil {
		t.Fatal(err)
	}
	e2 := u1.UUID
	a, err := s.Account().Find(e2)
	logrus.New().Info(err)
	assert.NoError(t, err)
	assert.NotNil(t, a)
}
\end{lstlisting}

\section{Вывод}

 