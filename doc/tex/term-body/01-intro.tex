\setcounter{page}{2}
\addchap{ВВЕДЕНИЕ}
В настоящее время социальные сети и форумы становятся частью повседневной жизни подавляющего количества пользователей сети Интернет. Активный обмен мнениями в сети привел к увеличению интереса к автоматическому извлечению тональности текста как со стороны научного сообщества, так и со стороны многих коммерческих организаций. 

Например, крупнейшие интернет-магазины, получают многочисленное
количество отзывов на свои товары. Выделение общих закономерностей в большом количестве неструктурированных текстов — это рутинная работа, требующая автоматизации. Таким образом, подбор эффективного алгоритма извлечения тональных оценок является актуальной задачей экономической и производственной аналитики.

Часто вся выводная часть анализа адекватности тональной разметки текста строится на сопоставлении частот, полученных по всему корпусу материалов, не учитывая региональные особенности. Получить статистические представления об оценке тональных прилагательных можно с помощью анкетирования -- метода проведения опроса, при котором для сбора сведений используется оформленный список вопросов.

Цель работы -- спроектировать и реализовать программное обеспечение для проведения анкетированного опроса экспертов о тональности прилагательных. Требуется учесть особенности модели данных, полученных при анкетировании при выборе базы данных.
\begin{itemize}
	\item проанализировать варианты хранения данных и выбрать подходящий вариант;
	\item спроектировать базу данных, описать ее сущности и связи;
	\item разработать систему разметки интерфейса;
	\item провести нагрузочное тестирование спроектированного приложения с учетом особенностей выбранной СУБД.
\end{itemize}

