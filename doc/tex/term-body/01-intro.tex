\setcounter{page}{2}
\addchap{ВВЕДЕНИЕ}
%Часто вся выводная часть анализа адекватности тональной разметки текста строится на сопоставлении частот, полученных по всему корпусу материалов, не учитывая региональные особенности.\cite{balance} Получить статистические представления об оценке тональных прилагательных можно с помощью анкетирования -- метода проведения опроса, при котором для сбора сведений используется оформленный список вопросов.
%
%Аналогичное исследование \cite{original} показывает, что жители Великобритании и Соединенных Штатов оценили тональные прилагательные по-разному: например, слово <<abysmal>> было оценено в Великобритании на 1.21 балла в среднем, в то время как в Соединенных Штатах на 2.55 -- разница в 1.34 балла.
%
%Составленный по данным, полученным в таком опросе словарь оценочной лексики будет показывать высокую точность в силу учета особенностей диалекта. 
%Цель работы -- спроектировать и реализовать программное обеспечение для проведения анкетированного опроса экспертов о тональности прилагательных. Для достижения поставленной цели требуется решить следующие задачи:
%\begin{itemize}
%	\item проанализировать варианты хранения данных и выбрать подходящий вариант;
%	\item спроектировать базу данных, описать ее сущности и связи;
%	\item разработать систему разметки интерфейса;
%	\item ....
%\end{itemize}

