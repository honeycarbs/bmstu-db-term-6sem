\chapter{Конструкторский раздел}\label{sec:design}
\section{\protect\justifying\protect\RaggedRight Проектирование отношений сущностей}
На рисунке \ref{fig:chen} приведена концептуальная схема проектируемой БД в нотации Чена. 
\begin{center}
	\begin{figure}[H]
		\centering
		\includegraphics[width=\linewidth]{assets/term-chen.drawio.png}
		\caption{ER-диаграмма сущностей базы данных в нотации Чена}
		\label{fig:chen}
	\end{figure}
\end{center}

Проектируемая база данных ориентирована на хранение информации, получаемой из web-приложения, содержащего систему оценки тональных прилагательных. Функционал приложения должен включать в себя регистрацию, авторизацию и возможность проставления оценок прилагательным. Соответственно, в базе данных можно выделить ряд сущностей. 
\begin{enumerate}
	\item Аккаунт хранит информацию, необходимую при регистрации -- электронная почта и пароль.
	\item Пользователь содержит информацию о личных характеристиках субъекта тональности -- национальность, возраст и гендерная принадлежность. 
	\item Населенный пункт -- это одна из характеристик субъекта тональности. Поскольку пользователь мог менять места проживания, связь этих двух сущностей -- <<многие ко многим>>.
	\item Место обучения -- сущность, содержащая характеристику об образовании, которое получает пользователь.
	\item Результат тестирования включает в себя оценку тонального прилагательного, присвоенная пользователем во время рабочей сессии. Возможные значения оценки -- числа от <<1>> до <<10>>.
	\item Тональное прилагательное.
\end{enumerate} 

\section{Ролевая модель} 
В ролевой модели\cite{Leonard1973} операции, которые необходимо выполнять в рамках какой-либо служебной обязанности пользователя системы, группируются в набор, называемый <<ролью>>.

При использовании ролевой политики управление доступом осуществляется в две стадии: для каждой роли указывается набор полномочий, представляющий набор прав доступа к объектам или частям приложения, затем каждому пользователю назначается его роль.

В разрабатываемой системе выделены три роли, каждой из которых соответствует конкретный функционал системы:
\begin{enumerate}
	\item Опрашиваемому -- функционал регистрации и авторизации, а также прохождения опроса и просмотр его результатов.
	\item Модератору -- функционал просмотра списка всех пользователей системы.
	\item Администратору -- функционал просмотра результатов опроса любого пользователя и общей статистики по опросу.
\end{enumerate}


\section{Нефункциональные характеристики графовых БД}
Зачастую в приложениях, активно использующих операции ввода-вывода, при выполнении одной бизнес-операции происходят массовое чтение и запись связанных между собой данных. В графовых базах данных выполняется несколько операций внутри логического подграфа общего набора данных. Подобное множество можно преобразовать в набор более крупных, тесно связанных операций.

В реляционных базах данных с увеличением размеров и количества данных начинают проявляться недостатки соединения таблиц и ухудшается производительность. Использование смежности без индексов позволяет графовым базам данных перемещаться по сложным соединениям в графе эффективнее, независимо от общего размера набора данных.
\subsection{Нерелевантность внешних кэшей}
Графовые базы данных имеют более высокую эффективность поиска за счет графовой организации хранения данных, благодаря этому разница в скорости доступа к данным в основной СУБД и внешних кэшах незначительна.\cite{Sholichah2020} 

В случае реляционных СУБД кэши используются для выделения <<горячих>>  данных и быстрого обращения к ним. Кэширующие базы данных как правило имеют тип <<ключ-значение>>, реализующие хеш-таблицу, в которой находится уникальный ключ и указатель на конкретный объект данных. Релеватность использования кэшей с реляционными СУБД достигается за счет не только меньшей ассимптотической сложности, но и хранения в данных в оперативной памяти вместо диска, используемого самой СУБД.\cite{aws} Таким образом, из общего набора данных выделяются <<горячие>>, скорость доступа к которым следует по возможности увеличивать. Ассимптотическая сложность алгоритма доступа к данным в графовых СУБД кэш-хранилищ совпадает.
К тому же, использование сторонних кэширующих хранилищ накладывает ограничения на размерность и связанность данных, достаточные, чтобы считать использование их лишенным смысла.

\section{\protect\justifying\protect\RaggedRight Проектирование клиентского приложения}
На рисунке \ref{fig:usecase} представлена диаграмма вариантов использования приложения. 
\begin{center}
	\begin{figure}[H]
		\centering
		\includegraphics[width=0.85\linewidth]{assets/term-uc.drawio.png}
		\caption{Диаграмма вариантов использования системы}
		\label{fig:usecase}
	\end{figure}
\end{center}
Пользователь системы, имеющий роль <<Модератор>>, может просматривать информацию о всех пользователях, сохраненных в базе данных. Интерфейс пользователя с ролью <<Администратор>> расширяет интерфейс для <<Модератора>> возможностью просматривать общую статистику опроса и результатов всех пользователей.

В случае посещения страниц, для которых у пользователя не соответствующая роль, в интерфейсе отображается соответствующее сообщение и доступ к запрашиваемому функционалу не предоставляется.

Интерфейс незарегистрированного пользователя включает возможность регистрироваться и осуществлять вход в систему. После регистрации или авторизации интерфейс расширяется: опрашиваемый может проходить тестирование и смотреть свои результаты. 

На рисунке \ref{fig:bpmn} представлена модель бизнес-процессов в нотации BPMN.  
\begin{center}
	\begin{figure}[H]
		\centering
		\includegraphics[width=\linewidth]{assets/term-bpmn.png}
		\caption{модель бизнес-процессов в нотации BPMN}
		\label{fig:bpmn}
	\end{figure}
\end{center}
Бизнес-процесс состоит одного пула <<пользователь>>, поделенного на три дорожки -- администратор, модератор и опрашиваемый. Дорожки определяются ролью пользователя, хранимой в базе данных. 

Опрашиваемый посещает сайт и проходит либо регистрацию, либо авторизацию в зависимости от того, был ли он уже зарегистрирован в системе. Далее ему предлагается пройти опрос либо просмотреть результаты в случае если опрос уже пройден. Результаты опроса после его заполнения отправляются в хранилище данных. 

Модератор имеет доступ к части хранилища с информацией о всех зарегистрированных пользователях и может просматривать информацию о любом пользователе. Администратор имеет доступ к результатам и может просматривать общую статистику опроса.

\addsec{Вывод}
Проектируемая база данных хранит информацию, получаемую из web-приложения, содержащего систему оценки тональных прилагательных. В базе выделен соответствующий ряд сущностей, приведенный на рисунке \ref{fig:chen}. Для базы выделено три роли: <<Опрашиваемый>>, <<Модератор>> и <<Администратор>>, каждая из которых соответствует дорожке пула <<пользователь>>, присутствующей в бизнес-процессе (рисунок \ref{fig:bpmn}). 

Использование сторонних кэширующих хранилищ накладывает ограничения на размерность и связанность данных, достаточные, чтобы считать использование их лишенным смысла -- графовые базы данных имеют более высокую эффективность поиска за счет графовой организации хранения данных.


